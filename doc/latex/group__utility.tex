\section{Utility}
\label{group__utility}\index{Utility@{Utility}}
\subsection*{Functions}
\begin{CompactItemize}
\item 
{\footnotesize template$<$typename Assignable1, typename Assignable2$>$ }\\\_\-\_\-host\_\-\_\- \_\-\_\-device\_\-\_\- void {\bf komrade::swap} (Assignable1 \&a, Assignable2 \&b)
\end{CompactItemize}


\subsection{Function Documentation}
\index{utility@{utility}!swap@{swap}}
\index{swap@{swap}!utility@{utility}}
\subsubsection[swap]{\setlength{\rightskip}{0pt plus 5cm}template$<$typename Assignable1, typename Assignable2$>$ \_\-\_\-host\_\-\_\- \_\-\_\-device\_\-\_\- void komrade::swap (Assignable1 \& {\em a}, \/  Assignable2 \& {\em b})\hspace{0.3cm}{\tt  [inline]}}\label{group__utility_g9b1d6e2dffcef4f1815ff019c7351c20}


{\tt swap} assigns the contents of {\tt a} to {\tt b} and the contents of {\tt b} to {\tt a}. This is used as a primitive operation by many other algorithms.

\begin{Desc}
\item[Parameters:]
\begin{description}
\item[{\em a}]The first value of interest. After completion, the value of b will be returned here. \item[{\em b}]The second value of interest. After completion, the value of a will be returned here.\end{description}
\end{Desc}
\begin{Desc}
\item[Template Parameters:]
\begin{description}
\item[{\em Assignable}]is a model of {\tt Assignable}.\end{description}
\end{Desc}
The following code snippet demonstrates how to use {\tt swap} to swap the contents of two variables.



\begin{Code}\begin{verbatim}  #include <komrade/utility.h>
  ...
  int x = 1;
  int y = 2;
  komrade::swap(x,h);

  // x == 2, y == 1
\end{verbatim}
\end{Code}

 